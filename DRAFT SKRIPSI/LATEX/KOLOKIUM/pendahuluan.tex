%----------------------------------------------------------------------------------------
%	PENDAHULUAN
%----------------------------------------------------------------------------------------
\section*{PENDAHULUAN} % Sub Judul PENDAHULUAN
% Tuliskan isi Pendahuluan di bagian bawah ini. 
% Jika ingin menambahkan Sub-Sub Judul lainnya, silakan melihat contoh yang ada.
% Sub-sub Judul 
\subsection*{Latar Belakang}
Indonesia memiliki lebih dari 32.000 spesies tumbuhan (\citeauthor{BAPPENAS2003} \cite*{BAPPENAS2003}). Dari kumpulan spesies tersebut terdapat tumbuhan obat di dalamnya. Menurut \citeauthor{ZUHUD2008} (\cite*{ZUHUD2008}) tidak kurang dari 2039 spesies tumbuhan obat berasal dari hutan Indonesia. Tanaman obat yang beraneka ragam jenis, habitus, dan khasiatnya mempunyai peluang besar serta memberi kontribusi bagi pembangunan dan pengembangan hutan (\citeauthor{HAMZARI2008} \cite*{HAMZARI2008}). Saat ini hutan Indonesia mengalami kerusakan dan kepunahan (\citeauthor{ZUHUD2008} \cite*{ZUHUD2008}). Oleh karena itu, diperlukan upaya untuk melestarikan tumbuhan obat. Salah satu cara untuk melestarikan tumbuhan obat adalah dengan cara mengenali tumbuhan obat (\citeauthor{HAMZARI2008} \cite*{HAMZARI2008}). \textit{Biodiversity Informatics} merupakan upaya untuk membuat sumber informasi keanekaragaman hayati global tersedia dalam format digital yang efisien, dan untuk mengembangkan alat yang efektif dalam menganalisis dan memahami data tersebut (\citeauthor{GILLMANE2009} \cite*{GILLMANE2009}). Informasi yang dapat diperoleh dari \textit{biodiversity informatics} adalah informasi mengenai taksonomi, gambar tumbuhan, lingkungan, dan DNA tumbuhan. 

Impementasi dari \textit{biodiversity informatics} sudah menghasilkan beberapa sistem yang menyediakan informasi mengenai tumbuhan.  Integrated Taxonomic Information System  (ITIS) dan Global Biodiversity Information Facility (GBIF) menyediakan informasi yang luas tentang tumbuhan. Sistem tersebut dibuat dengan menggunakan model basis data relasional. Model basis data relasional menimbulkan masalah apabila digunakan pada sistem berbasis inferensi (\citeauthor{LAALLAM2013} \cite*{LAALLAM2013}). Selain itu model basis data relasional dapat menghasilkan data yang berganda. Oleh sebab itu, dibutuhkan pemodelan data yang dapat mengatasi hal tersebut. Metode pemodelan data yang dapat menangani sistem berbasis inferensi adalah ontologi.

Ontologi adalah metode yang digunakan untuk merepresentasikan ide, fakta dan lain sebagainya, yang digunakan untuk mendefinisikan hubungan dan klasifikasi dari pengetahuan tertentu (\citeauthor{JEPSEN2010} \cite*{JEPSEN2010}). Ontologi dapat menentukan kelas, hubungan, fungsi dan objek lain (\citeauthor{DILECCE2008} \cite*{DILECCE2008}). Selain itu, model ontologi lebih sesuai diterapkan pada web semantik dibandingkan dengan model basis data relasional (\citeauthor{LAALLAM2013} \cite*{LAALLAM2013}).

Penelitian dengan menggunakan ontologi mengenai tumbuhan obat sudah banyak dilakukan, seperti penelitian tentang ontologi yang digunakan untuk menganalisis hubungan tumbuhan obat dengan istilah medis yang standar (\citeauthor{VADIVU2012} \cite*{VADIVU2012}). Penelitian yang terkait dengan ontologi gen juga sudah pernah dilakukan untuk menghasilkan data gen yang dinamis dan terkontrol (\citeauthor{ASHBURNERM2000} \cite*{ASHBURNERM2000}). Namun, penelitian tersebut belum menghasilkan hubungan antara tumbuhan obat dengan infomasi gen-nya. Pada penelitian ini akan dibuat sistem yang memanfaatkan web semantik yang digunakan untuk mengintegrasikan informasi gen dengan tumbuhan obat. \textit{Resource Description Framework} (RDF) akan diterapkan pada sistem ini untuk mengatasi masalah integrasi dengan data tumbuhan obat. RDF merupakan standar untuk merepresentasikan data yang berbentuk grafik dan membagikan dengan manusia dan mesin. 

Berdasarkan latar belakang di atas penelitian ini akan mengembangkan sistem web semantik yang memberikan informasi gen tumbuhan obat. Selain itu sistem web semantik ini akan menyediakan \textit{web service} yang memungkinkan untuk terintegrasi dengan sistem web semantik yang lain.

% Sub-sub Judul 
\subsection*{Tujuan}
Tujuan yang ingin dicapai dalam penelitian ini adalah:
\begin{enumerate}[noitemsep] 
\item Menerapkan teknologi web semantik untuk menggunakan ontologi gen (\textit{gene ontology}).
\item Mengintegrasikan sistem ontologi gen dengan data tumbuhan obat.
\item Mengintegrasikan sistem ontologi gen dengan sistem yang lain, yaitu ontologi tumbuhan (\textit{plant ontology}) dan ontologi lingkungan (\textit{environment ontoloy}).
\end{enumerate}

\subsection*{Ruang Lingkup}
Ruang lingkup penelitian adalah:
\begin{enumerate}[noitemsep] 
\item Ontologi yang digunakan dalam penelitian ini berasal dari situs geneontology.org.
\item Data korpus yang digunakan terbatas hanya tumbuhan obat. 
\item Sistem ontologi gen yang dibangun diintegrasikan dengan ontologi tanaman dan ontologi lingkungan.
\end{enumerate}

\subsection*{Manfaat}
Hasil pengembangan sistem ontologi gen ini diharapkan dapat membantu memberikan informasi mengenai gen dari tumbuhan obat.
