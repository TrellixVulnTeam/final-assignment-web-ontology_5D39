%----------------------------------------------------------------------------------------
%	PENDAHULUAN
%----------------------------------------------------------------------------------------
\section*{PENDAHULUAN} % Sub Judul PENDAHULUAN
% Tuliskan isi Pendahuluan di bagian bawah ini. 
% Jika ingin menambahkan Sub-Sub Judul lainnya, silakan melihat contoh yang ada.
% Sub-sub Judul 
\subsection*{Latar Belakang}
Indonesia memiliki lebih dari 32.000 spesies tumbuhan (\citeauthor{BAPPENAS2003} \cite*{BAPPENAS2003}). Saat ini hutan Indonesia mengalami kerusakan dan kepunahan (\citeauthor{ZUHUD2008} \cite*{ZUHUD2008}). Oleh karena itu, diperlukan upaya untuk melestarikan tumbuhan. Salah satu cara untuk melestarikan tumbuhan adalah dengan cara mengenali tumbuhan tersebut. \textit{Biodiversity Informatics} merupakan upaya untuk membuat sumber informasi keanekaragaman hayati global tersedia dalam format digital yang efisien, dan untuk mengembangkan alat yang efektif dalam menganalisis dan memahami data tersebut (\citeauthor{GILLMANE2009} \cite*{GILLMANE2009}). Informasi yang dapat diperoleh dari \textit{biodiversity informatics} adalah informasi mengenai taksonomi, gambar tumbuhan, lingkungan, dan DNA tumbuhan. 

Impementasi dari \textit{biodiversity informatics} sudah menghasilkan beberapa sistem yang menyediakan informasi mengenai tumbuhan.  Integrated Taxonomic Information System  (ITIS) dan Global Biodiversity Information Facility (GBIF) menyediakan informasi yang luas tentang tumbuhan. Proses identifikasi dan pengelolaan informasi keanekaragaman hayati tersebut memerlukan sistem yang terpadu dan holistic dengan menggunakan IPTEKS komputer yang berkembang pesat saat ini (\citeauthor{HERDIYENI2013} \cite*{HERDIYENI2013}). Upaya pemanfaatan IPTEKS yang telah dilakukan seperti diantaranya pembangunan sistem IPB Biodiversity Informatics (IPBiotics) untuk pengelolaan informasi keanekaragaman hayati sumber daya alam Indonesia. Sistem BI tersebut berguna meningkatkan pengelolaan pengetahuan (knowledge management), eksplorasi, analisis, sintesis dan interpretasi data keanekaragaman hayati mulai dari level genomik, level spesies sampai dengan level ekosistem (\citeauthor{HERDIYENI2013} \cite*{HERDIYENI2013}). Pada pengembangan selanjutnya, sistem IPBiotics didesain agar dapat melakukan inferensi pengetahuan. Sistem yang ada saat ini masih menggunakan model basis data relasional. Permasalahannya adalah model basis data relasional kurang sesuai diterapkan pada sistem berbasis inferensi (\citeauthor{LAALLAM2013} \cite*{LAALLAM2013}).  

Ontologi adalah metode yang digunakan untuk merepresentasikan ide, fakta dan lain sebagainya, yang digunakan untuk mendefinisikan hubungan dan klasifikasi dari pengetahuan tertentu (\citeauthor{JEPSEN2010} \cite*{JEPSEN2010}). Ontologi dapat menentukan kelas, hubungan, fungsi dan objek lain (\citeauthor{DILECCE2008} \cite*{DILECCE2008}). Selain itu, model ontologi lebih sesuai diterapkan pada web semantik dibandingkan dengan model basis data relasional (\citeauthor{LAALLAM2013} \cite*{LAALLAM2013}).

Penelitian dengan menggunakan ontologi mengenai tumbuhan sudah banyak dilakukan, seperti penelitian tentang ontologi yang digunakan untuk menganalisis hubungan tumbuhan obat dengan istilah medis yang standar (\citeauthor{VADIVU2012} \cite*{VADIVU2012}). Penelitian yang terkait dengan ontologi gen juga sudah pernah dilakukan untuk menghasilkan data gen yang dinamis dan terkontrol (\citeauthor{ASHBURNERM2000} \cite*{ASHBURNERM2000}) dan pemodelan ontologi tumbuhan obat menggunakan pengetahuan etnobotani (Sanjaya 2014). Namun penelitian tersebut berfokus pada pemodelan ontologi. Berdasarkan latar belakang di atas penelitian ini akan membangun sistem web semantik yang memanfaatkan ontologi yang sudah ada untuk mengelola informasi tumbuhan.

% Sub-sub Judul 
\subsection*{Tujuan}
Tujuan yang ingin dicapai dalam penelitian ini adalah:
\begin{enumerate}[noitemsep] 
\item Membangun sistem \textit{biodiversity informatics} tumbuhan menggunakan ontologi gen (\textit{gene ontology}).
\item Menerapkan sistem inferensi pengetahuan pada ontologi gen untuk mengembalikan informasi berupa \textit{molecular function}, \textit{biological processes} dan \textit{cellular components} yang terdapat pada tumbuhan.
\end{enumerate}

\subsection*{Ruang Lingkup}
Ruang lingkup penelitian adalah:
\begin{enumerate}[noitemsep] 
\item Ontologi yang digunakan dalam penelitian ini berasal dari situs geneontology.org.
\item Membangun \textit{biodiversity informatics} pada level genetik.
\end{enumerate}

\subsection*{Manfaat}
Manfaat yang diinginkan dari penelitian ini adalah membantu proses dokumentasi data dan pengetahuan keanekaragaman hayati tumbuhan. Dengan memanfaatkan ontologi, memungkinkan sistem untuk melakukan inferensi dan mengembalikan informasi detail tumbuhan mengenai \textit{molecular function}, \textit{biological processes} dan \textit{cellular components} tumbuhan. Dengan informasi tersebut, diharapkan proses dokumentasi keanekaragaman hayati tumbuhan berjalan lebih efektif.
