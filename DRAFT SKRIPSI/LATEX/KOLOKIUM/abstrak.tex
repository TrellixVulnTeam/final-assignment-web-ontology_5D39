%----------------------------------------------------------------------------------------
%	ABSTRACT
%----------------------------------------------------------------------------------------
\Abstract{\scriptsize 
% ---- Tuliskan abstrak di bagian ini seperti contoh.
Indonesia memiliki lebih dari 32.000 spesies tumbuhan. Dari kumpulan spesies tersebut terdapat tumbuhan obat di dalamnya. Tidak kurang dari 2039 spesies tumbuhan obat berasal dari hutan Indonesia. Saat ini hutan Indonesia mengalami kerusakan dan kepunahan. Oleh karena itu, diperlukan upaya untuk melestarikan tumbuhan obat. Salah satu cara untuk melestarikan tumbuhan obat adalah dengan cara mengenali tumbuhan obat.  Informasi yang dibutuhkan mengenai tumbuhan obat sulit untuk ditemukan.  Berdasarkan hal tersebut maka muncul bidang baru dalam pengumpulan informasi tumbuhan yang bernama \textit{biodiversity informatics}. Metode pemodelan data yang dapat menangani sistem berbasis inferensi adalah ontologi. Ontologi dapat diterapkan pada web semantik. Penelitian ini akan mengembangkan sistem web semantik yang memberikan informasi genetika tumbuhan obat. Selain itu sistem web semantik ini akan menyediakan \textit{web service} yang memungkinkan untuk terintegrasi dengan sistem web semantik yang lain.
% ---- Akhir bagian abstrak
\normalsize}
